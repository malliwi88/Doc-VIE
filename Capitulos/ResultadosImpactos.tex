\section{Resultados Esperados o Productos}
\label{sec:Resultados Esperados o Productos}

Los resultados de esta investigación, buscan construir argumentos y conclusiones suficientes para determinar si la técnica de posicionamiento propuesta es una alternativa de solución a la problemática de precisión en posicionamiento de dispositivos GPS inmersos en ambientes urbanos. Y con los cuales se pueda dar respuesta a la pregunta de investigación bajo la cual se origina %originó 
este trabajo de investigación.\\

\textbf{Conducentes al fortalecimiento de la capacidad científica nacional}:

\begin{itemize}
	\item Tesis de maestría que aporta al avance del país en una línea de investigación sistemas de posicionamiento, promoviendo el uso de plataformas y librerias para la adquisición y procesamiento de datos GPS/GNSS.
	\item El trabajo aporta cuerpo del conocimiento para futuros desarrollos y aplicaciones en el uso de sistemas de posicionamiento, para desarrollos tecnológicos que permitan la transformación de la cuidad de Bucaramanga en una ciudad inteligente.
\end{itemize}


\textbf{Relacionados con la generación de conocimiento y/o nuevos desarrollos tecnológicos}:


\begin{itemize}
	\item Estudio de los algoritmos del estado del arte en técnicas de posicionamiento GPS/GNSS.
	\item Desarrollo de una técnica de posicionamiento basada en la interacción de dispositivos GPS inmersos en ambientes urbanos.
	\item Publicación de artículo en revista científica relacionada con el área de interés del proyecto, en un plazo máximo de un año después de culminado el proyecto.
\end{itemize}



\section{Conformación y trayectoria del grupo de investigación}
\label{sec:Trajectoria SC3}

\textbf{Lineas de Investigación de SC3}:

\begin{itemize}
	\item Internet de las cosas (IoT).
	\item Supercómputo y cálculo científico.
	\item Computación distribuida.
	\item GPS, GNSS y Sistemas de posicionamiento satelital.
	\item Sistema Embebidos.
	%\item La red de interconexión entre los equipos...
\end{itemize}

\textbf{Proyectos desarrollados por SC3}:

\begin{itemize}
	\item ARQUITECTURA E IMPLEMENTACION DE UN SISTEMA DE ADQUISICION DE DATOS PARA SISTEMAS GLOBALES DE NAVEGACION POR SATELITE (GNSS), Diego Fernando Acosta Ortiz, Raul Ramos Pollan, Santiago Soley Rimblas. Trabajo de Grado \textbf{S 30343}.
\end{itemize}
