\chapter{OBJETIVOS} 
\label{sec:objetivos}

Los objetivos planteados en este apartado tienen como propósito identificar los aspectos que permitan dar respuesta a la pregunta, ¿Es posible mejorar el nivel de precisión en localización dentro de ambientes urbanos, mediante la interacción de dispositivos de posicionamiento GPS inmersos en este ambiente?

\section{General}
Plantear una técnica de posicionamiento basada en la interacción de dispositivos GPS, para determinar si el intercambio de información entre dispositivos permite mejorar en el nivel de precisión dentro de ambientes urbanos.

% Plantear un procedimiento Diseñar una técnica de posicionamiento basado en interacción de dispositivos de posicionamiento, que contribuya a la mejora en el nivel de precisión en tareas de localización dentro de ambientes urbanos.\\


\section{Específicos}

Para alcanzar el objetivo general planteado en este trabajo de investigación, se considera importante el desarrollo de las siguientes actividades:

\begin{itemize}

\item Plantear un modelo matemático que permita construir una técnica de posicionamiento basada interacción de dispositivos GPS.

\item Validar el planteamiento matemático involucrado en la técnica de posicionamiento mediante simulación, para saber si los resultados obtenidos concuerdan con el objetivo de la investigación.

\item Seleccionar un mecanismo de interacción entre dispositivos gps, basado en la información requerida por el modelo matemático para realizar el cálculo de posicionamiento.

\item Evaluar el nivel de precisión alcanzado por los dispositivos dentro de ambientes urbanos, para identificar los casos de uso de la técnica de posicionamiento planteada.

\end{itemize}



