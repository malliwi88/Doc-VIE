\thispagestyle{empty}

%\chapter{Resumen}
%\section{Resumen}

\begin{center}
\Large\centering\bfseries
\textsc{RESUMEN}
\end{center}

Tras la ratificación de la eliminación de la característica de disponibilidad selectiva en futuras constelaciones GPS en 2007 %\footnote{\url{http://www.gps.gov/systems/gps/modernization/sa/}}
, se abrió paso a que los sistemas de posicionamiento satelital puedan alcansar niveles de precisión de hasta un orden de 1 metro (1m) o menos.\\

En busca de este objetivo Europa en 2011, decide hacer el lanzamiento de un nuevo sistema GNSS independiente, con capacidad de cálculo y corrección activa de orbitas y capacidad de interacción con los dos sistemas ya existentes, GLONASS(Rusia) y GPS(USA); pudiendo conformar una constelación de satélites de al menos 86 satélites funcionales en total para el 2020. Sin embargo, uno de los principales problemas de los sistemas de navegación basados en GNSS es la degradación de la precisión cuando alguno de los satélites es bloqueado, como es el caso de los edificios dentro del ambiente urbano, la vegetación espesa en zona selvática o fenómenos de interferencia entre las señales que llegan al receptor.\\

Por otra parte, se ha venido observando que a medida que la cantidad de satelites disponibles aumenta la ocurrencia de fallos en la sincronización de reloj también aumenta, afectando el nivel de precisión en posicionamiento de los receptores GNSS. El segmento de control GNSS es el encargado de monitorear y marcar los satelites que presentan errores de sincronización o estan defasados de orbitas; de esta forma estos satélites no son aptos para una buena precisión en posicionamiento.\\
% Basado en las patentes 
%	http://www.google.com/patents/US20140232595
%	https://www.google.ch/patents/US9170109

En este orden de ideas, las técnicas de posicionamiento muestran ser muy robustas en condiciones de estáticas y con buena visibilidad, la navegación dentro de las ciudades impone unas condiciones diferentes, bajo las cuales solo un monto de todos los satélites es visible. Por ello, los estudios mas recientes en sistemas de posicionamiento satelital, exploran el desarrollo de algoritmos para la detección de fallos y/o técnicas de posicionamiento para su uso dentro de ambientes urbanos, con el objetivo de alcanzar mayores niveles de precisión en posicionamiento.\\

%Una buena parte de los reportes de literatura durante la ultima década han sido enfocados a caracterizar la precisión de GPS y GNSS en ambientes a cielo abierto y urbanos, empleando dispositivos de posicionamiento estáticos. En en estos reportes se hace énfasis en el impacto que tienen la baja visibilidad de satelitales, interferencia inter-canal y el obstrucción de señales, como los factores que impiden alcanzar mayores niveles de precisión en dispositivos GPS inmersos en cañones urbanos.\\ 

%Cuando se tiene en cuenta que para 2020 habrá una gran
Teniendo en cuenta la infraestructura satelital disponible y la oferta de tecnologías y dispositivos para la recepción de señales GPS en el mercado de dispositivos electrónicos actual; surge la inquietud que que da origen a la presente investigación: \\

\textit{¿Es posible contribuir a la mejorar del nivel de precisión en posicionamiento dentro de ambientes urbanos, mediante una técnica de posicionamiento apoyada en la interacción de dispositivos GPS? y ¿Que niveles de precisión podría alcanzar esta técnica?}\\

%El contenido de este documento, es el resultado inicial del trabajo investigativo y metodológico que busca alternativas de respuesta a la pregunta de investigación planteada.\\

%El contenido de este documento, es el resultado de la inicial que tienen como objetivo el establecimiento de las bases metodológicas necesarias que permitan dar respuesta a la pregunta de investigación planteada con anterioridad.\\

%\vspace{-0.5cm}
\textbf{Key-words}:
DGPS, GNSS, GPS, Cooperative-GPS. 
\\


%\section{Presentación del documento} 

%Hasta este punto, lo que queda por resaltar de este breve recorrido histórico, es que el objetivo de llevar a cabo tareas de posicionamiento con alta precisión, es en si la premisa fundamental que argumenta el porqué de la existencia de los sistemas de posicionamiento satelital.\\

%\begin{shaded}
%El presente documento está organizado de la siguiente manera:\\
%\end{shaded}

%Primero se lleva a cabo la introducción y breve contextualización del proyecto en la sección~\ref{sec:introduccion}. La sección  \ref{sec:estadodelarte} presenta en estado del arte involucrado con la planificaci\'on de tareas sobre plataformas de cómputo de alto rendimiento y la problem\'atica del consumo energ\'etido en este tipo de arquitecturas. Luego en la secci\'on ~\ref{sec:propuesta}, se presenta de forma mas detallada la problemática, la justifcación y motivación de que sustenta la presente propuesta de investigación. En la sección~\ref{sec:objetivos} de se presenta los objetivos, alcances de la propuesta, al igual que los productos esperados de la misma.\\

%Finalmente las secciones~\ref{sec:metodologia}, ~\ref{sec:cronograma}, ~\ref{sec:Presupuesto}, presenta la metodología y cronograma de actividades el planteados para el desarrollo de la investigación junto al presupuesto necesario para el desarrollo, con lo cual se concluye el contenido de este documento de propuesta de investigación.
