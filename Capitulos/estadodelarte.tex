\chapter{Estado del Arte}
\label{sec:estadodelarte}

En calidad de abordar la búsqueda de precisión decimétrica en sistemas de posicionamiento, se ha realizado un breve recorrido por la literatura para clasificar aquellos trabajos que se consideran relevantes para el desarrollo de la presente investigación.\\ 

Se encuentra que muchas de las técnicas revisadas, han sido planteadas para ambientes a cielo abierto y empleadas en dispositivos de posicionamiento estáticos. Adicionalmente, los estudios desarrollados dentro de ambientes urbanos y con receptores GPS en movimiento experimentan dos situaciones especiales, la baja visibilidad de satélites y el alto efecto multipath producto de la inmersión en cañones urbanos.\\

Producto de la revisión de los trabajos desarrollados durante las últimas dos décadas y relacionados con temas de interés para le presente investigación, se considera la clasificación de los mismos en 3 grandes grupos. \\

En el primer grupo a considerar, se encuentran los trabajos relacionados con el modelado matemático y representación de fenómenos relacionados con el comportamiento de la ionosfera, troposfera durante el viaje de la señal desde el satélite al receptor, así como el planteamiento de modelos corrección y predicción de orbita de satélites y/o errores por efectos de reflexión y difracción.\\

Dentro de este primer grupo, se pueden considerar los estudios del fenómeno multipath; como es el caso del trabajo de Xia\cite{Xia_2004}, quien presenta un breve estado del arte trabajos relacionados con el modelado y mitigación de los errores causados por Multipath, desde el punto de vista de hardware y software. Wiell\cite{Weill} igualmente, hace una reseña de las técnicas que pueden ser empleadas para la mitigación del efecto multipath, bien sea con el uso de tipos específicos de antenas y/o configuraciones de las mismas, así como también el uso correlaciones entre las señales con vista directa e indirecta a satélites, para la identificación y la supresión de las señales contaminadas con efecto multipath.\\

Para la mitigación de efecto multipath basados en configuraciones de antenas; la publicación de \cite{AlJewari}, presenta un estudio en el que se consigue caracterizar una configuración de dos antenas para recepción de señales GPS que alcanza valores de aproximadamente 1.5 para HDOP, con lo cual un receptor GPS estaría en capacidad de recibir señales directas en indirectas que permitirían alcanzar mayores niveles de precisión en posicionamiento.\\

Otros estudios, direccionados a la caracterización y descripción de los fenómenos de propagación de señales dentro de los ambientes urbanos y sub-urbanos han sido presentados por \cite{lehner2005novel}, \cite{steingass2008differences}; de los cuales se resalta que las principales problemáticas en este tipo de ambientes son reflexiones con alta potencia de señal, escasa visibilidad de satélites.% y dinámica de receptor GPS. 
En cuanto al impacto de estas situaciones en la precisión del posicionamiento, se menciona que generan perdida del enganche de reloj, saltos de ciclo(cycle slips), dificultan la convergencia y resolución de ambigüedad y por ende causan aumento del error en el posicionamiento. \\

La definición de modelos ionosféricos y troposféricos, algoritmos para la detección de saltos de ciclo han sido abarcados en los trabajos de Klobuchar \cite{Klobuchar_1987} y Zhang\cite{Zhang_2006} respectivamente.\\

Un segundo grupo, es el orientado a la mejora y desarrollo de aplicaciones GNSS apoyado en la integración de tecnologías emergentes y la tecnología GNSS, para aplicaciones civiles, comerciales y científicas, como son los casos de integración de unidades de medición inercial (IMU) con GPS, Indoor-GPS, Outdoor-GPS, etc, en sistemas de transporte inteligente (ITS), medición del desplazamiento de capas tectónicas, medición de nivel de cauce en ríos y lagos, entre otros.\\

Si bien es cierto que existen muchos trabajo relacionados con el desarrollo de modelos matemáticos y técnicas de posicionamiento, estas propuestas algunas veces aumentan el costo económico o computacional de las tareas de posicionamiento, impidiendo la masificación de sistemas de posicionamiento de bajo costo y mayor precisión; como las principales condiciones requeridas para nuevas aplicaciones dentro del marco de referencia de las ciudades Inteligentes (Smart Cities), como es el caso del documento presentado por \cite{barreca2010future}, en el que se exponen las oportunidades del sistema de posicionamiento Galileo para el desarrollo de soluciones en seguridad y autenticación apoyadas en tecnología GNSS.\\

Un estudio interesante sobre la movilidad de la población dentro de los mercados ambulantes de Hong Kong, es desarrollado por \cite{Tsang_2011}; el propósito del estudio es caracterizar la importancia de los mercados ambulantes para los visitantes de la ciudad, para así poder promover políticas de seguridad a través de la recomendación de mercados ambulantes seguros para los turistas. Este es un caso de aplicación bastante tratado para el futuro de las ciudades inteligentes al rededor del mundo.\\

El estudio de \cite{Meijles_2014}, enfocado al estudio del patrones de movimiento en parques naturales dentro de ambientes urbanos. Meijles apoya la importancia de su estudio en la necesidad de visualizar la creciente demanda de usuarios de los parques naturales en ciudades y sitios apartados, ya que es evidente que se puede saturar la capacidad ecológica de los mismos generando daños irreversibles, categorizando su aporte como una herramienta para el manejo y cuidado de parques. La relevancia de este trabajo para con el objetivo de esta propuesta, es la conclusión presentada por Meijles, en la que resalta la importancia de la precisión para el desarrollo de su estudio, sin embargo hace énfasis en la limitante que impone la precisión (10-15m) en los dispositivos GPS actuales, ya que induce un nivel de incertidumbre al determinar la posición de los individuos dentro y fuera de los limites permitidos de los parques naturales. Un estudio similar es presentado por \cite{Orellana_2012}.\\

Tang \cite{Tang_2014} expone el desarrollo de un sistemas de posicionamiento cooperativos para sistemas de trasporte inteligente(ITS), en el cual menciona que el posicionamiento relativo juega un papel importante en vehículos autónomos, ya que en escenarios urbanos el posicionamiento absoluto de los vehículos no juega un papel tan crucial, cuando el objetivo es evitar colisiones entre ellos. Tang presenta un estudio de correlación entre el fenómeno multipath y el error en el pseudo-rango dentro de cañones urbanos, del cual concluye que la caracterización de la dinámica del error asociado al multipath, permite reducir el error en posicionamiento de vehículos en movimiento.\\

Y finalmente un tercer grupo, tiene como objeto el abordar las etapas de pre-procesamiento y pos-procesamiento de señales de radiofrecuencia, con la ayuda de funciones de software que tienen como única propósito, reducir la complejidad del hardware en los receptores GNNS, disminuir el costo de implementación de las soluciones basadas en posicionamiento satelital, permitiendo la especialización de los componentes de hardware que integran la etapa de adquisición y unidades de procesamiento integradas en dispositivos GPS, mientras las tareas de procesamiento son delegadas a rutinas de software apoyadas en los fundamentos del procesamiento digital de señales.\\

Los estudios relacionados con el planteamiento de técnicas de posicionamiento, podría considerarse como una clasificación intermedia entre el segundo y el tercer grupo, en la cual se busca proponer modelos matemáticos mediante de rutinas de software, que pueden acceder a información adicional si lo requieren para realizar correcciones en el cálculo de posicionamiento.\\

Los estudios reportados en literatura acerca de técnicas de posicionamiento basados en trabajo colaborativo de sensores GPS, se encuentra el caso del trabajo de Berefelt y Boberg\cite{berefelt2004collaborative}, quienes reportan resultados de simulación de tres técnicas de navegación colaborativa en ambientes urbanos, de los cuales se resalta el uso de la una técnica denominada "Relative Vector" en la cuál un dispositivo de ubicación y coordenadas conocidas es empleado por un segundo dispositivo como punto de referencia para determinar su posición, mitigando el impacto de la poca visibilidad de satélites y efecto multipath.\\

Finalmente \cite{Li_2014}, presenta una interesante visión del futuro de la tierra basada el uso de Internet de la cosas (IOT). En en trabajo de Li se propone una completa infraestructura denominada GEOSS o sistema de sistemas de observación de la tierra, la cual esta apoyada en dispositivos móviles que trabajan como sensores y entregan información a un servicio web encargado de  generar modelos de información global, con lo cuales los científicos podrían analizar con mayor facilidad la superficie de la tierra. La información presentada mediante modelos de información 3D, tendrían información útil para la prevención de desastres naturales, pronostico del clima y estudios de fenómenos geológicos como desplazamientos de capas tectónicas. Li al final de su documento, plantea un servicio de posicionamiento apoyado en la nube de información, con la cual los dispositivos de posicionamiento satelital podrían mejorar la precisión en posicionamiento.\\  

Todas las áreas presentadas en esta clasificación pueden considerarse que tienen de fondo un objetivo común, y es que con el aporte de cada uno de ellos lo que se busca es tener un posicionamiento de precisión. \\





