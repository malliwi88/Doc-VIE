\chapter{Introducción}
\label{sec:introduccion}

Desde sus principios, los sistemas de posicionamiento satelital han sido construidos con el único propósito de contar con una herramienta tecnológica que permita localizar un dispositivo GPS en casi cualquier lugar del mundo, en cualquier momento.\\ 

Fue así como desde sus inicios en 1960, el sistema de posicionamiento global GPS creado por el DOD\footnote{DoD: Departamento de estado de los Estados Unidos de Norteamérica}, fue enfocado a tareas de alta precisión como herramienta estratégica para tareas militares, desde la precisión en aplicaciones GPS juega un papel importante.\\

Las bondades de GPS abrían paso a un sin numero de aplicaciones civiles y comerciales que no daban espera, como la búsqueda de mejoras en la navegación marítima y las dificultades propias del aterrizaje en aviones comerciales de la época. No fue sino hasta los 80s que el sistema GPS estuvo disponible para aplicaciones de uso civil. Sin embargo, la apertura del servicio de posicionamiento estándar (SPS) estuvo sujeto a una manipulación de las señales públicas para degradar la precisión, las cuales fueron argumentadas por el (DoD) como medidas preventivas de seguridad nacional. De forma que, aunque los usuarios podían acceder a un servicio de posicionamiento libre y gratuito, estaban sujetos a la limitación técnica impuesta por la Disponibilidad Selectiva (SA o "Negación de posición"), que impedia alcanzar niveles de precisión con incertidumbre inferior a 100m en condiciones óptimas de operación, en presencia de SA.\\ 

Hasta este punto el aspecto importante a resaltar es, que el objetivo de llevar a cabo tareas de posicionamiento con alta precisión, es en sí, la premisa que argumenta el porqué de la existencia de los sistemas de posicionamiento satelital.\\ 

Tras la eliminación de la característica de Disponibilidad selectiva en futuras constelaciones GPS en el año 2007 \footnote{\url{http://www.gps.gov/systems/gps/modernization/sa/}}, se abrió paso a que el esfuerzo de los investigadores fuese dedicado al estudio de fenómenos propios involucrados en el funcionamiento de las señales satelitales, con el único propósito de reducir el nivel de precisión hasta el orden de 1 metro (1m).\\

Fue entonces como en 2011, Europa decide hacer el lanzamiento de un nuevo sistema GNSS independiente, como una herramienta estratégica para el avance tecnología y comercial de ese continente. El sistema de posicionamiento satelital GALILEO como se le conoce, contará con capacidad de cálculo y corrección activa de orbitas y capacidad de interacción con los dos sistemas ya existentes, GLONASS(Rusia)\footnote{GLONASS \url{https://www.glonass-iac.ru/en/GLONASS/}} y GPS(USA)\footnote{GPS \url{https://www.glonass-iac.ru/en/GPS/}}, pudiendo conformar una constelación de satélites de al menos 86 satélites funcionales en total para el 2020 \footnote{GALILEO \url{http://gpsworld.com/directions-2016-galileo-strategic-tool-for-european-autonomy/}}; con los cuales se daría solución a integridad, disponibilidad, continuidad y precisión en sistemas y aplicaciones GNSS.\\

Ahora bien, con toda esta infraestructura satelital disponible, ¿que es lo que preocupa a la comunidad científica involucrada con los sistemas de posicionamiento?. Podría decir que los trabajos de investigación de la última década involucrados con sistemas de posicionamiento están orientados en 3 grandes grupos. \\

En el primer grupo de la clasificación a considerar, están los trabajos relacionados con el modelado matemático %para la representación 
de fenómenos relacionados con el comportamiento de la ionosfera, troposfera durante el viaje de la señal desde el satélite al receptor, así como el planteamiento de modelos corrección y predicción de orbita de satélites además de modelados de los efectos de reflexión y difracción de las señales satelitales. \\

Un segundo grupo, es el orientado a la mejora y desarrollo de aplicaciones GNSS apoyado en la integración de nuevas tecnologías, para el uso masivo de tecnología GNSS en aplicaciones civiles, comerciales y científicas; como son los casos de integración de unidades de medición inercial (IMU) con GPS, Indoor-GPS, Outdoor-GPS, etc; en sistemas de transporte inteligente (ITS), medición del desplazamiento de capas tectónicas, medición de nivel de cauce en ríos y lagos, entre otros.\\

Y finalmente un tercer grupo, tiene como objeto el abordar las etapas de pre-procesamiento y pos-procesamiento de señales de radiofrecuencia, con la ayuda de funciones de software, que tienen como única propósito, reducir la complejidad del hardware en los receptores GNNS, disminuir el costo de implementación de las soluciones basadas en posicionamiento satelital y finalmente, abrir paso a la especialización de los componentes de hardware que integran la etapa de adquisición y unidades de cómputo, mientras las tareas de procesamiento son delegadas a rutinas de software apoyadas en los fundamentos del procesamiento digital de señales.\\